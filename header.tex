%
% Project header
%

\documentclass[
12pt, 		% 12pt font size
pdftex, 	% we are just using pdf
a4paper,	% a4 paper format
titlepage,% title has its own page
oneside,	% single-page view
%twocolumn,% use two text columns
headsepline, % hr after headline
footsepline, % hr before footer,
bibliography=totoc, % Include literaturverzeichnis in inhaltsverzeichnis
listoftables=totoc,
ngerman,	% it's a german document
%draft			% create debug build. remove for release build
]{scrbook}

% Load project configuration
%
% Project configuration
%

% Document title
\newcommand{\cfgDocTitle}{Eine wissenschaftliche Arbeit}
\newcommand{\cfgDocSubTitle}{Mit LaTeX}

% Document classification
\newcommand{\cfgDocClassification}{Hausarbeit}

% Document version
\newcommand{\cfgDocVersion}{1.0}

% Last modification date
\newcommand{\cfgDateLastModification}{15. November 2021}

%%%%%%%%%%%%%%%%%%%%%%%%%%%%%%%%%%%%%%%%%%%%%%%%%%%%

% Author name
\newcommand{\cfgAuthorName}{Mux Mastermann}

% Author matriculation number
\newcommand{\cfgAuthorMatriculationNum}{696969\_CHANE\_THAT!!}

% Author contact
\newcommand{\cfgAuthorContact}{mux@hotmail.com}

% Author city
\newcommand{\cfgAuthorCity}{Worms}

%%%%%%%%%%%%%%%%%%%%%%%%%%%%%%%%%%%%%%%%%%%%%%%%%%%

% University city
\newcommand{\cfgUniversityCity}{Worms}

% University name
\newcommand{\cfgUniversityName}{Hochschule \cfgUniversityCity}

% University department
\newcommand{\cfgUniversityDepartment}{Fachbereich Informatik}

% University degree course
\newcommand{\cfgUniversityDegreeCourse}{Angewandte Informatik - dual (B.Sc)}

% University supervisor name
\newcommand{\cfgUniversitySupervisorName}{Michael D.Werle-Rutter}

% Semester year
\newcommand{\cfgSemesterYear}{2021/22}

% Semester name
\newcommand{\cfgSemesterName}{Wintersemester \cfgSemesterYear}

% University course name
\newcommand{\cfgCourseName}{Wissenschaftliches Arbeiten}


\parindent=0pt % disable paragraph indent

\usepackage{ngerman}
\usepackage[utf8]{inputenc}
\usepackage[T1]{fontenc}

% Paket um Grafiken im Dokument einbetten zu k�nnen.
% Im PDF sind GIF, PNG, und PDF Grafiken m�glich.
\usepackage{graphicx}

% caption setzt per default hypcap=true, sodass Sprungmarken im PDF-File nicht nur 
% zu den Captions von Bildern springen, sondern an den oberen Rand der Bilder.
\usepackage{caption}

% Font 'Latin Modern Family' verwenden.
% Verwende dieses Paket wenn du DML selbst kompilierst.
\usepackage{lmodern}

% Tabellen
\usepackage{array}

% Mathematische Formeln
\usepackage{amsmath, amsthm, amssymb, mathtools}

% Deutsche Anführungszeichen
\usepackage[babel, german=quotes]{csquotes}

% Mehrere Bilder als eine Abbildung
\usepackage{subcaption}

% Seiten-bottom-padding
\usepackage[bottom=2.5cm]{geometry}

% Dashed lines
\usepackage{arydshln}

% Colored text
\usepackage{color}

% Custom title format
\usepackage{sectsty}
\allsectionsfont{\rmfamily}
\chaptertitlefont{\rmfamily\hspace*{0.5em}}

% Custom side marings
\usepackage{geometry}

% Zeilenabstand 1.5
\usepackage[onehalfspacing]{setspace}

% Tab-respecting code printing
\usepackage{fancyvrb}

% Abkürzungsverzeichnis
\usepackage{acronym}

% Appendix
\usepackage[toc,page]{appendix}
\renewcommand{\appendixname}{Anhang}
\renewcommand{\appendixtocname}{Anhang}
\renewcommand{\appendixpagename}{Anhang}

% Metadaten
\usepackage[pdftex,
    pdfauthor={\cfgAuthorName},
    pdftitle={\cfgDocTitle},
    pdfsubject={\cfgDocSubTitle},
    pdfproducer={Latex},
    pdfcreator={pdflatex}]{hyperref}

% Load custom macros
%
% Custom macros
%

% Will encapsule a text in italic << >>
\newcommand{\enpointy}[1]{\emph{\textlangle\textlangle{#1}\textrangle\textrangle}}

% Will add a vertical line to tables
\newcommand{\tblVline}{\vline\hspace{2mm}}

% Will do a full reference, with both a number AND a caption. Like '1.2.3 Coding the code'
\newcommand{\fullref}[1]{\ref{#1} \nameref{#1}}


% Load custom environments
%
% Custom environments
%

% Code frame
\DefineVerbatimEnvironment{code}{Verbatim}
{
    tabsize=4,
    samepage=true,
    frame=lines,
    numbers=left,
}

% Nonpagebreaking chapter
\newenvironment{npbrChap}
{
    \begingroup
    \let\clearpage\relax
}
{
    \endgroup
}

% Nicely placed image environment
% Features:
%   Margin-top
%   placed exactly where it's put in the source code
%   image and caption always stays on same page
\newenvironment{nicepic}
{
    \begin{minipage}{\textwidth}
    \vspace{7.5mm}
    \begin{center}
}
{
    \end{center}
    \end{minipage}
}

